\setlength{\mathindent}{0.5cm}

Gesucht ist die letzte Nichtnull-Ziffer $LZ(n!)$ einer Zahl $n \in \mathbb{N}$. \\
Wir suchen anhand des Beispiels $n=20$ eine andere Darstellungsweise für $n!$:
\begin{equation*}
	\begin{split}
		20! &= 20\cdot19\cdot18\cdot17\cdot16\cdot15\cdot14\cdot13\cdot12\cdot11\cdot10\cdot9\cdot8\cdot7\cdot6\cdot5\cdot4\cdot3\cdot2\cdot1 \\
		&= 20\cdot15\cdot10\cdot5 \;\cdot\; 19\cdot18\cdot17\cdot16\cdot14\cdot13\cdot12\cdot11\cdot9\cdot8\cdot7\cdot6\cdot4\cdot3\cdot2\cdot1 \\
		&= (5\cdot4)(5\cdot3)(5\cdot2)(5) \cdot 19\cdot18\cdot17\cdot16\cdot14\cdot13\cdot12\cdot11\cdot9\cdot8\cdot7\cdot6\cdot4\cdot3\cdot2\cdot1 \\
		&= 5^4 \cdot 4! \;\cdot\; 19\cdot18\cdot17\cdot16\cdot14\cdot13\cdot12\cdot11\cdot9\cdot8\cdot7\cdot6\cdot4\cdot3\cdot2\cdot1 \\
		&= 5^4 \cdot 4! \,\cdot\, 19(9\cdot2)17(8\cdot2)\,\cdot\,(7\cdot2)13(6\cdot2)11\,\cdot\,9(4\cdot2)7(3\cdot2)\,\cdot\,(2\cdot2)3(1\cdot2)1 \\
		&= 5^4 \cdot 4! \cdot 2^8 \cdot (19\cdot9\cdot7\cdot8)(7\cdot13\cdot6\cdot11)(9\cdot4\cdot7\cdot3)(2\cdot3\cdot1\cdot1)
	\end{split}
\end{equation*}

Mithilfe dieser Darstellung lässt sich die gesuchte Ziffer bestimmen:
\begin{equation*}
	\begin{split}
		LZ(20!) &= LZ\left(5^4 \cdot 4! \cdot 2^8 \cdot (19\cdot9\cdot7\cdot8)(7\cdot13\cdot6\cdot11)(9\cdot4\cdot7\cdot3)(2\cdot3\cdot1\cdot1)\right) \\
		&= LZ\left(10^4 \cdot 2^4 \cdot 4! \cdot (19\cdot9\cdot7\cdot8)(7\cdot13\cdot6\cdot11)(9\cdot4\cdot7\cdot3)(2\cdot3\cdot1\cdot1)\right) \\
	\end{split}
\end{equation*}

Für $LZ(n!)$ spielt der Faktor $10^4$ keine Rolle, da nur die letzte Nichtnull-Ziffer von Interesse ist, er kann also weggelassen werden:
\begin{equation*}
	\begin{split}
		LZ(20!) &= LZ\left(2^4 \cdot 4! \cdot (19\cdot9\cdot7\cdot8)(7\cdot13\cdot6\cdot11)(9\cdot4\cdot7\cdot3)(2\cdot3\cdot1\cdot1)\right)
	\end{split}
\end{equation*}

Die letzte Stelle der 4er-Gruppen an Faktoren am Ende kann auch bestimmt werden, hierfür gibt es zwei Fälle zu betrachten:
\begin{multicols}{2}
	\noindent
	\begin{equation*}
		\begin{split}
			&LZ\left(\dots9 \cdot \frac{\dots8}{2} \cdot \dots7 \cdot \frac{\dots6}{2}\right) \\
			&= LZ\left(9\cdot\frac{8}{2}\cdot7\cdot\frac{6}{2}\right) \\
			&= LZ\left(756\right) \\
			&= 6
		\end{split}
	\end{equation*}
	\begin{equation*}
		\begin{split}
			&LZ\left(\frac{\dots4}{2} \cdot \dots3 \cdot \frac{\dots2}{2} \cdot \dots1\right) \\
			&= LZ\left(\frac{4}{2} \cdot 3 \cdot \frac{2}{2} \cdot 1\right) \\
			&= 6
		\end{split}
	\end{equation*}
\end{multicols}

\say{$\mathit{\dots}$} steht hierbei für beliebig viele Ziffern (innerhalb einer 4er-Gruppe sind diese immer gleich).
\\[10pt]
Hierbei spielt es keine Rolle, ob die Ziffern $\dots$ vor den Paaren $8$ und $6$ bzw. $4$ und $2$ gerade oder ungerade sind, da das Produkt des jeweiligen Paars immer auf die selbe Ziffer endet:
\begin{multicols}{2}
	\noindent
	\begin{equation*}
		\begin{split}
			\frac{08}{2} \cdot \frac{06}{2} &= 1\underline{2} \\
			\frac{18}{2} \cdot \frac{16}{2} &= 7\underline{2} \\
			\frac{28}{2} \cdot \frac{26}{2} &= 18\underline{2} \\
			\vdots
		\end{split}
	\end{equation*}
	\begin{equation*}
		\begin{split}
			\frac{04}{2} \cdot \frac{02}{2} &= \underline{2} \\
			\frac{14}{2} \cdot \frac{12}{2} &= 4\underline{2} \\
			\frac{24}{2} \cdot \frac{22}{2} &= 13\underline{2} \\
			\vdots
		\end{split}
	\end{equation*}
\end{multicols}

In beiden Fällen endet eine Gruppe an 4 Faktoren immer auf die Ziffer $6$, also kann der Term von oben weiter vereinfacht werden:
\begin{equation*}
	\begin{split}
		LZ(20!) &= LZ\left(2^4 \cdot 4! \cdot 6^4\right) \\
		&= LZ\left(12^4 \cdot 4!\right)
	\end{split}
\end{equation*}

Die $12$ kann zu einer $2$ umgeschrieben werden, da nur die letzte Stelle von Interesse ist:
\begin{equation*}
	\begin{split}
		LZ(20!) &= LZ\left(2^4 \cdot 4!\right) \\
		LZ(n!) &= LZ\left(2^m \cdot m!\right) \text{ für } m = \left\lfloor\frac{n}{5}\right\rfloor
	\end{split}
\end{equation*}

Für ein $n$, für das $n\bmod 5 \neq 0$ gilt, wird für die Berechnung oben genannter Faktoren der Operator $\lfloor\;\rfloor$ benötigt: $\lfloor n \rfloor$ ist die größte ganze Zahl $\leq n$.
\\[10pt]
Für solche $n$ gilt:
\begin{equation*}
	\begin{split}
		LZ(n!) &= LZ\left(2^m \cdot m! \cdot \frac{n!}{(m\cdot5)!} \right) \text{ für } m = \left\lfloor\frac{n}{5}\right\rfloor
	\end{split}
\end{equation*}

Der letzte Faktor stellt dabei die noch fehlenden Faktoren $\leq n$ und $> m\cdot5$ dar. Beispielsweise für $n=23$ steht der letzte Faktor für $23\cdot22\cdot21$.
\\[10pt]
Für diese Formel muss immer $n\geq5$ gelten, da sie sich sonst mit $m=0$ zu $LZ(n!)$ kürzen würde und für den Beweis nicht hilfreich wäre.
\\[10pt]
Durch den Faktor $2$ als Teil von $2^m$ kann $LZ(n!)$ also nur gerade Werte annehmen. Die Ziffern $1$, $3$, $5$, $7$ und $9$ können also aus der Folge ausgeschlossen werden ($0$ ist durch die Aufgabenstellung ja auch ausgeschlossen). Es bleiben also noch die Ziffern $2$, $4$, $6$ und $8$.
\\[10pt]
Nun wird der Faktor $2^m$ genauer betrachtet:

\begin{table}[h!]
	\hspace{0.5cm}\begin{tabular}{l|c|c|c|c|c|c|c|c|c|c}
		m & 1 & 2 & 3 & 4 & 5 & 6 & 7 & 8 & 9 & \dots \\
		\hline
		$2^m$ & \underline{2} & \underline{4} & \underline{8} & 1\underline{6} & 3\underline{2} & 6\underline{4} & 12\underline{8} & 25\underline{6} & 51\underline{2} & \dots \\
	\end{tabular}
\end{table}

$LZ\left(\frac{n!}{(m\cdot5)!}\right)$ nimmt für $k$ = $n \bmod 10$ den $k.$ Wert der folgenden Reihe an: $1;1;2;6;4; 1;6;2;6;4 $
\\
\begin{minipage}{0.5\linewidth}
	\begin{alignat*}{3}
		\frac{\dots9!}{\dots5!} &= \dots9 \cdot \dots8 \cdot \dots7 \cdot \dots6 &= 302\underline{4} \\[10pt]
		\frac{\dots8!}{\dots5!} &= \dots8 \cdot \dots7 \cdot \dots6 &= 33\underline{6} \\[10pt]
		\frac{\dots7!}{\dots5!} &= \dots7 \cdot \dots6 &= 4\underline{2} \\[10pt]
		\frac{\dots6!}{\dots5!} &= \dots6 &= \underline{6} \\[10pt]
		\frac{\dots5!}{\dots5!} & &= \underline{1}
	\end{alignat*}
\end{minipage}\begin{minipage}{0.5\linewidth}
	\begin{alignat*}{3}
		\frac{\dots4!}{\dots0!} &= \dots4 \cdot \dots3 \cdot \dots2 \cdot \dots1 &= 2\underline{4} \\[10pt]
		\frac{\dots3!}{\dots0!} &= \dots3 \cdot \dots2 \cdot \dots1 &= \underline{6} \\[10pt]
		\frac{\dots2!}{\dots0!} &= \dots2 \cdot \dots1 &= \underline{2} \\[10pt]
		\frac{\dots1!}{\dots0!} &= \dots1 &= \underline{1} \\[10pt]
		\frac{\dots0!}{\dots0!} &= \dots1 &= \underline{1}
	\end{alignat*}
\end{minipage}

In diesen beiden Faktoren lässt sich für immer größer werdende $n$ und damit auch $m$ jeweils ein sich unendlich oft wiederholendes iteratives Muster erkennen. Folglich kommt in dem Gesamtprodukt jede der möglichen Ziffern $2$, $4$, $6$ und $8$ unendlich oft als letzte Nichtnull-Ziffer vor. \qed
