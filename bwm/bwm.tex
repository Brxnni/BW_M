%!TEX root = bwm.tex
% Standardzeug
\documentclass[12pt,a4paper,oneside]{article}
\usepackage[
	includehead,
	nomarginpar,
	top=1cm,
	inner=1.5cm,
	outer=5cm
]{geometry}
\usepackage[utf8]{inputenc}
\usepackage[T1]{fontenc}

\usepackage[ngerman]{babel}
\usepackage[
    left = \glqq{},%
    right = \grqq{}%
]{dirtytalk}

% Coole font
\usepackage{palatino}

% Schöner Burgundfarbton
\usepackage{xcolor}
\definecolor{darkburgundy}{HTML}{721C33}
\definecolor{burgundy}{HTML}{B32A4D}

% Zwischenüberschriften
% Keine Nummerierung
\setcounter{secnumdepth}{0}
% In Burgund
\usepackage{titlesec}
\titleformat*{\section}{\LARGE\bfseries\color{burgundy}}
\titleformat*{\subsection}{\Large\bfseries\color{darkburgundy}}

% Für Unteraufgaben, auch in Burgund
\usepackage{enumitem}
\setlist[itemize]{
	leftmargin=*,
	noitemsep,
	wide=0pt,
	font=\bf\color{darkburgundy}
}

% Keine Absatzindentierung
\setlength{\parindent}{0pt}

% Grafiken für A3 und A4
\usepackage{tikz}
\usetikzlibrary{
	angles,
	quotes,
	positioning,
	patterns.meta, % Linienmuster
	decorations.pathreplacing, % Geschweifte Klammern
	babel % Kompatibilität mit Babel (Anführungszeichen werden standardmäßig überschrieben)
}

% Mathezeug
\usepackage[fleqn]{amsmath}
\usepackage{amssymb}
\usepackage{amsthm}
% Für °
\usepackage{siunitx}

\usepackage{multicol}

% Fancyheader sind praktisch
\usepackage{fancyhdr}
\setlength{\headheight}{15pt}

\begin{document}

% Nummerierte Seiten, auf jeder Seite müssen Namen stehen
\pagestyle{fancy}
\fancyhead[L]{\thepage}
\fancyhead[R]{Henry Feuerlein, Arne Stein}
\fancyfoot[]{}

\section{Aufgabe 1}
\setlength{\mathindent}{2cm}

Gegeben seien die Werte $i_1$ bis $i_9$, auf welche die Positionen der Felder, $1$ bis $9$, zufällig verteilt sind.
Fridolin springt von Position $0$ zu $i_1$, dann zu $i_2$ usw. bis $i_9$ und schließlich zurück zu $0$. \\[10pt]
Für eine so gegebene beliebige Reihenfolge der Zahlen $1$ bis $9$ lässt sich die zurückgelegte Strecke mit der folgenden Formel berechnen:
\begin{align*}
	d &= i_1 + i_9 + \sum_{k=1}^{8} |i_{k+1} - i_k|
\end{align*}
\Fburg{a)} Ein Beispiel für eine Sprungfolge mit der Gesamtdistanz $20$ ist \\
$1, 2, 3, 4, 5, 6, 8, 7, 9$ (gefunden durch Ausprobieren):
\begin{align*}
	d &= 1 + 9 + |2-1|+|3-2|+|4-3|+|5-4| \\
	&+ |6-5|+|8-6|+|7-8|+|9-7| \\
	&= 20
\end{align*}

Um zu beweisen, dass für die Gesamtdistanz keine ungerade Strecke herauskommen kann, wird die Parität gebildet:
\begin{align*}
	d \bmod 2 &= \left(i_1 + i_9 + \sum_{k=1}^{8} |i_{k+1} - i_k|\right) \bmod 2
\end{align*}

Die Betragsstriche können ignoriert werden, da sie an der Parität der Differenz $i_{k+1}-i_k$ nichts ändern:
\begin{align*}
	d &\equiv i_1 + i_9 + \sum_{k=1}^{8} i_{k+1} - i_k \pmod 2 \\
	&\equiv i_1 + i_9 + (i_2 - i_1) + (i_3 - i_2) + (i_4 - i_3) + (i_5 - i_4) \\
	&+ (i_6 - i_5) + (i_7 - i_6) + (i_8 - i_7) + (i_9 - i_8) \pmod 2 \\[10pt]
	&\equiv i_1 - i_1 + i_2 - i_2 + i_3 - i_3 + i_4 - i_4 + i_5 - i_5 + i_6 - i_6 \\
	&+ i_7 - i_7 + i_8 - i_8 + i_9 + i_9 \pmod 2 \\[10pt]
	&\equiv 2 i_9 \pmod 2
\end{align*}

$2 i_9$ ist immer gerade, also gilt:
\begin{align*}
	d \equiv 2 i_9 \equiv 0 \pmod 2
\end{align*}

Folglich ist $d$ für alle Sprungreihenfolgen eine gerade Zahl.
\\
\Fburg{b)} Die zurückgelegte Strecke kann also nie $25$ Längeneinheiten betragen. \qed

\pagebreak

\section{Aufgabe 2}
\setlength{\mathindent}{0.5cm}

Gesucht ist die letzte Nichtnull-Ziffer $LZ(n!)$ einer Zahl $n \in \mathbb{N}$. \\
Wir suchen anhand des Beispiels $n=20$ eine andere Darstellungsweise für $n!$:
\begin{equation*}
	\begin{split}
		20! &= 20\cdot19\cdot18\cdot17\cdot16\cdot15\cdot14\cdot13\cdot12\cdot11\cdot10\cdot9\cdot8\cdot7\cdot6\cdot5\cdot4\cdot3\cdot2\cdot1 \\
		&= 20\cdot15\cdot10\cdot5 \;\cdot\; 19\cdot18\cdot17\cdot16\cdot14\cdot13\cdot12\cdot11\cdot9\cdot8\cdot7\cdot6\cdot4\cdot3\cdot2\cdot1 \\
		&= (5\cdot4)(5\cdot3)(5\cdot2)(5) \cdot 19\cdot18\cdot17\cdot16\cdot14\cdot13\cdot12\cdot11\cdot9\cdot8\cdot7\cdot6\cdot4\cdot3\cdot2\cdot1 \\
		&= 5^4 \cdot 4! \;\cdot\; 19\cdot18\cdot17\cdot16\cdot14\cdot13\cdot12\cdot11\cdot9\cdot8\cdot7\cdot6\cdot4\cdot3\cdot2\cdot1 \\
		&= 5^4 \cdot 4! \,\cdot\, 19(9\cdot2)17(8\cdot2)\,\cdot\,(7\cdot2)13(6\cdot2)11\,\cdot\,9(4\cdot2)7(3\cdot2)\,\cdot\,(2\cdot2)3(1\cdot2)1 \\
		&= 5^4 \cdot 4! \cdot 2^8 \cdot (19\cdot9\cdot7\cdot8)(7\cdot13\cdot6\cdot11)(9\cdot4\cdot7\cdot3)(2\cdot3\cdot1\cdot1)
	\end{split}
\end{equation*}

Mithilfe dieser Darstellung lässt sich die gesuchte Ziffer bestimmen:
\begin{equation*}
	\begin{split}
		LZ(20!) &= LZ\left(5^4 \cdot 4! \cdot 2^8 \cdot (19\cdot9\cdot7\cdot8)(7\cdot13\cdot6\cdot11)(9\cdot4\cdot7\cdot3)(2\cdot3\cdot1\cdot1)\right) \\
		&= LZ\left(10^4 \cdot 2^4 \cdot 4! \cdot (19\cdot9\cdot7\cdot8)(7\cdot13\cdot6\cdot11)(9\cdot4\cdot7\cdot3)(2\cdot3\cdot1\cdot1)\right) \\
	\end{split}
\end{equation*}

Für $LZ(n!)$ spielt der Faktor $10^4$ keine Rolle, da nur die letzte Nichtnull-Ziffer von Interesse ist, er kann also weggelassen werden:
\begin{equation*}
	\begin{split}
		LZ(20!) &= LZ\left(2^4 \cdot 4! \cdot (19\cdot9\cdot7\cdot8)(7\cdot13\cdot6\cdot11)(9\cdot4\cdot7\cdot3)(2\cdot3\cdot1\cdot1)\right)
	\end{split}
\end{equation*}

Die letzte Stelle der 4er-Gruppen an Faktoren am Ende kann auch bestimmt werden, hierfür gibt es zwei Fälle zu betrachten:
\begin{multicols}{2}
	\noindent
	\begin{equation*}
		\begin{split}
			&LZ\left(\dots9 \cdot \frac{\dots8}{2} \cdot \dots7 \cdot \frac{\dots6}{2}\right) \\
			&= LZ\left(9\cdot\frac{8}{2}\cdot7\cdot\frac{6}{2}\right) \\
			&= LZ\left(756\right) \\
			&= 6
		\end{split}
	\end{equation*}
	\begin{equation*}
		\begin{split}
			&LZ\left(\frac{\dots4}{2} \cdot \dots3 \cdot \frac{\dots2}{2} \cdot \dots1\right) \\
			&= LZ\left(\frac{4}{2} \cdot 3 \cdot \frac{2}{2} \cdot 1\right) \\
			&= 6
		\end{split}
	\end{equation*}
\end{multicols}

\say{$\mathit{\dots}$} steht hierbei für beliebig viele Ziffern (innerhalb einer 4er-Gruppe sind diese immer gleich).
\\[10pt]
Hierbei spielt es keine Rolle, ob die Ziffern $\dots$ vor den Paaren $8$ und $6$ bzw. $4$ und $2$ gerade oder ungerade sind, da das Produkt des jeweiligen Paars immer auf die selbe Ziffer endet:
\begin{multicols}{2}
	\noindent
	\begin{equation*}
		\begin{split}
			\frac{08}{2} \cdot \frac{06}{2} &= 1\underline{2} \\
			\frac{18}{2} \cdot \frac{16}{2} &= 7\underline{2} \\
			\frac{28}{2} \cdot \frac{26}{2} &= 18\underline{2} \\
			\vdots
		\end{split}
	\end{equation*}
	\begin{equation*}
		\begin{split}
			\frac{04}{2} \cdot \frac{02}{2} &= \underline{2} \\
			\frac{14}{2} \cdot \frac{12}{2} &= 4\underline{2} \\
			\frac{24}{2} \cdot \frac{22}{2} &= 13\underline{2} \\
			\vdots
		\end{split}
	\end{equation*}
\end{multicols}

In beiden Fällen endet eine Gruppe an 4 Faktoren immer auf die Ziffer $6$, also kann der Term von oben weiter vereinfacht werden:
\begin{equation*}
	\begin{split}
		LZ(20!) &= LZ\left(2^4 \cdot 4! \cdot 6^4\right) \\
		&= LZ\left(12^4 \cdot 4!\right)
	\end{split}
\end{equation*}

Die $12$ kann zu einer $2$ umgeschrieben werden, da nur die letzte Stelle von Interesse ist:
\begin{equation*}
	\begin{split}
		LZ(20!) &= LZ\left(2^4 \cdot 4!\right) \\
		LZ(n!) &= LZ\left(2^m \cdot m!\right) \text{ für } m = \left\lfloor\frac{n}{5}\right\rfloor
	\end{split}
\end{equation*}

Für ein $n$, für das $n\bmod 5 \neq 0$ gilt, wird für die Berechnung oben genannter Faktoren der Operator $\lfloor\;\rfloor$ benötigt: $\lfloor n \rfloor$ ist die größte ganze Zahl $\leq n$.
\\[10pt]
Für solche $n$ gilt:
\begin{equation*}
	\begin{split}
		LZ(n!) &= LZ\left(2^m \cdot m! \cdot \frac{n!}{(m\cdot5)!} \right) \text{ für } m = \left\lfloor\frac{n}{5}\right\rfloor
	\end{split}
\end{equation*}

Der letzte Faktor stellt dabei die noch fehlenden Faktoren $\leq n$ und $> m\cdot5$ dar. Beispielsweise für $n=23$ steht der letzte Faktor für $23\cdot22\cdot21$.
\\[10pt]
Für diese Formel muss immer $n\geq5$ gelten, da sie sich sonst mit $m=0$ zu $LZ(n!)$ kürzen würde und für den Beweis nicht hilfreich wäre.
\\[10pt]
Durch den Faktor $2$ als Teil von $2^m$ kann $LZ(n!)$ also nur gerade Werte annehmen. Die Ziffern $1$, $3$, $5$, $7$ und $9$ können also aus der Folge ausgeschlossen werden ($0$ ist durch die Aufgabenstellung ja auch ausgeschlossen). Es bleiben also noch die Ziffern $2$, $4$, $6$ und $8$.
\\[10pt]
Nun wird der Faktor $2^m$ genauer betrachtet:

\begin{table}[h!]
	\hspace{0.5cm}\begin{tabular}{l|c|c|c|c|c|c|c|c|c|c}
		m & 1 & 2 & 3 & 4 & 5 & 6 & 7 & 8 & 9 & \dots \\
		\hline
		$2^m$ & \underline{2} & \underline{4} & \underline{8} & 1\underline{6} & 3\underline{2} & 6\underline{4} & 12\underline{8} & 25\underline{6} & 51\underline{2} & \dots \\
	\end{tabular}
\end{table}

$\frac{n!}{(m\cdot5)!}$ kann nur folgende Werte annehmen:
\begin{equation*}
	\begin{split}
		\dots9 \cdot \dots8 \cdot \dots7 \cdot \dots6 &= 302\underline{4} \\
		\dots8 \cdot \dots7 \cdot \dots6 &= 33\underline{6} \\
		\dots7 \cdot \dots6 &= 4\underline{2} \\
		\dots6 &= \underline{6} \\
		\dots4 \cdot \dots3 \cdot \dots2 \cdot \dots1 &= 1\underline{2} \\
		\dots3 \cdot \dots2 \cdot \dots1 &= \underline{6} \\
		\dots2 \cdot \dots1 &= \underline{2} \\
		\dots1  &= \underline{1}
	\end{split}
\end{equation*}

In diesen beiden Faktoren lässt sich für immer größer werdende $n$ und damit auch $m$ jeweils ein sich unendlich oft wiederholendes iteratives Muster erkennen. Folglich kommt in dem Gesamtprodukt jede der möglichen Ziffern $2$, $4$, $6$ und $8$ unendlich oft als letzte Nichtnull-Ziffer vor. \qed

\pagebreak

\section{Aufgabe 3}
\setlength{\mathindent}{4cm}

Um zu beweisen, dass $AE$ senkrecht zu $CD$ steht, werden die Steigungen beider Geraden in Abhängigkeit des Winkels $\beta$, der die Position von P auf dem Halbkreisbogen $AB$ bestimmt, berechnet.
\\[10pt]
(Für die gesamte Herleitung beziehen sich $\sin$ und $\cos$ immer nur auf die Variable unmittelbar danach, also nur $\alpha$ oder $\beta$.)

\subsection[]{Herleitung von $m_{AE}$}
Hierfür relevante Punkte und Strecken:

% Was häufig benutzt wird
\newcommand{\Koordinaten}{
	\coordinate (m) at (0,0);
	\coordinate (a) at (-1,0);
	\coordinate (b) at (1,0);
	\coordinate (c) at (-0.5,0.866);
	\coordinate (p) at (0.5,0.866);
	\coordinate (e) at (0.866,0.5);
}
\newcommand{\Halbkreis}{
	\begin{scope}
		\clip (a) rectangle (1,1);
		\draw[black] (m) circle(1);
		\draw[black] (a) -- (b);
	\end{scope}
}

\newcommand{\sina}{\sin\alpha}
\newcommand{\cosa}{\cos\alpha}
\newcommand{\sinb}{\sin\beta}
\newcommand{\cosb}{\cos\beta}

\begin{center}
	\begin{tikzpicture}[scale=4]
		% Koordinaten
		\Koordinaten
		\coordinate (kam) at (0.57732,0.33335);

		\Halbkreis
		% Kreis k_a
		\draw (kam) circle(0.33334);

		% Punkte
		\filldraw[black] (m) circle (0.5pt) node[anchor=north]{$M$};
		\filldraw[black] (a) circle (0.5pt) node[anchor=north]{$A$};
		\filldraw[black] (b) circle (0.5pt) node[anchor=north]{$B$};
		\filldraw[black] (p) circle (0.5pt) node[anchor=south west]{$P$};
		\filldraw[black] (e) circle (0.5pt) node[anchor=south west]{$E$};
		\filldraw[blue] (kam) circle (0.35pt);

		% Linien
		\draw[black] (a) -- (e);
		\draw[black] (m) -- (p);
		\draw[magenta, line width=1pt] (m) -- (e) node[midway,above]{$R$};
		\draw[magenta, line width=1pt] (a) -- (m) node[midway,below]{$R$};
		\draw[blue] (kam) -- (e) node[midway,below right]{$r$};
		\draw[blue] (kam) -- (0.57732,0) node[midway,right]{$r$};
		\draw[orange, line width=1pt] (e) -- (0.866,0) node[midway,left]{$h$};
		\draw[teal, line width=1pt] (b) -- (0.866,0) node[midway,below]{$x$};

		% Winkel
		\pic [draw,"$\beta$",angle eccentricity=1.7]{angle = b--m--e};
		\pic [draw,"$\beta$",angle eccentricity=1.7]{angle = e--m--p};
	\end{tikzpicture}
\end{center}
% diese erklärung ist nicht gut das muss besser irgendwie besser gehen (auch egal weil es die beim bwm eh nicht zu sehen bekommen, gibt glaub ich auch punktabzug :c)
Die Strecke $ME$ halbiert exakt den Kreissektor, weil der Mittelpunkt des Kreises $k_a$ exakt zwischen $M$ und $E$ liegt, also auf $ME$. Das liegt wiederum daran, dass $k_a$ maximal groß sein soll, aber $MB$ und $MP$ in genau einem Punkt berühren soll. Um maximal groß zu sein, \say{wächst} er also symmetrisch zu der Winkelhalbierenden in seinem Kreissektor, also symmetrisch zu der Strecke $ME$.
\begin{align*}
	m_{AE} &= \frac{\Delta y}{\Delta x} = \frac{h}{2R-x} \\
	\frac{h}{R} = \frac{r}{R-r} \leftrightarrow h &= \frac{Rr}{R-r} \tag{1}
\end{align*}

\begin{align*}
	\sinb &= \frac{\text{Gegenkathete}}{\text{Hypotenuse}} \\
	\sinb &= \frac{r}{R-r} \\
	r &= R \sinb - r \sinb \\
	r + r \sinb &= R \sinb \\
	r(1+\sinb) &= R \sinb \\
	r &= \frac{R \sinb}{1 + \sinb} \tag{2}
\end{align*}

\pagebreak
\begin{center}
	\begin{tikzpicture}[scale=13]
		% Punkte
		\Koordinaten
		\coordinate (b2) at (0.93969,0.34202);
		\coordinate (p2) at (0.98481,0.17365);
		\coordinate (p2_) at (0.98481,0);
		\coordinate (kam) at (0.83896,0.14772);
		\coordinate (kam_side) at (0.98481,0.14772);

		\filldraw[black] (m) circle (0.25pt) node[anchor=east]{$M$};
		\filldraw[black] (p2) circle (0.25pt) node[anchor=south west]{$P$};

		% Kreissektor mit 20°
		\draw[black] (m) -- (0:1) arc(0:20:1) -- cycle;
		% Kreis k_a
		\draw (kam) circle(0.14812);

		% Linien
		\draw[orange, line width=1pt] (p2) -- (0.98481,0) node[midway,right]{$h$};
		\draw[magenta, line width=1pt] (m) -- (b2) node[midway,above left]{$R$};
		\draw[violet] (kam) -- (p2) node[midway,above]{$r$};
		\draw[violet, line width=1pt] (kam) -- (0.83896,0) node[midway,left]{$r$};
		\draw[blue, line width=1pt] (m) -- (p2_) node[midway,below]{$R_0$};
		\draw[teal, line width=1pt] (p2_) -- (b) node[midway,below]{$x$};
		\draw[cyan, line width=1pt] (kam) -- (kam_side) node[midway,below]{$b$};
		\draw[black] (m) -- (p2);

		% Winkel
		\pic [draw,"$\beta$",angle eccentricity=2.6]{angle = b--m--b2};
		\pic [draw,"\tiny$\beta$\normalsize",angle eccentricity=2]{angle = kam_side--kam--p2};
	\end{tikzpicture}
\end{center}

\begin{align*}
	\frac{R_0}{h} &= \frac{R_0-b}{r} \\
	r R_0 &= h R_0 - hb \\
	r R_0 - h R_0 &= -hb \\
	R_0 (r-h) &= -hb \\
	R_0 &= \frac{-hb}{r-h}
\end{align*}
Mit $ \cosb = \frac{b}{r} \Leftrightarrow b = r \cosb $ und $(1)$:
\begin{align*}
	R_0 &= \frac{\frac{-Rr}{R-r}*r \cosb}{r-\frac{Rr}{R-r}} \\
	&= \frac{\left(\frac{-Rr}{R-r}*r \cosb\right)}{\left(\frac{(R-r)r}{R-r}-\frac{Rr}{R-r}\right)} \\
	&= \frac{\left(\frac{-Rr}{R-r}*r \cosb\right)}{\left(\frac{Rr-r^2-Rr}{R-r}\right)} \\
	&= \frac{\left(\frac{-Rr^2 \cosb}{R-r}\right)}{\left(-\frac{r^2}{R-r}\right)} \\
	&= \frac{-Rr^2 \cosb * (R-r)}{-(R-r)r^2} \\
	R_0 &= R \cosb
\end{align*}

\begin{align*}
	R_0 + x &= R \\
	x &= R - R_0 \\
	&= R - R \cosb \tag{3}
\end{align*}

\begin{samepage}
	\begin{align*}
		m_{AE} &= \frac{h}{2R - x} \\
	\end{align*}
	Mit $(1)$ und $(3)$: \nopagebreak
	\begin{align*}
		m_{AE} &= \frac{\frac{Rr}{R-r}}{2R - (R-R \cosb)} \\
		&= \frac{\frac{Rr}{R-r}}{R+R \cosb} \\
		&= \frac{Rr}{(R-r)(R+R \cosb)} \\
		&= \frac{r}{(R-r)(1+\cosb)}
	\end{align*}
	Mit $(2)$: \nopagebreak
	\begin{align*}
		m_{AE} &= \frac{\frac{R \sinb}{1+\sinb}}{(R-r)(1+\cosb)} \\
		&= \frac{R \sinb}{(R-r)(1+\cosb)(1+\sinb)} \\
		&= \frac{R \sinb}{(R-\frac{R \sinb}{1+\sinb})(1+\cosb)(1+\sinb)} \\
		&= \frac{R \sinb}{(\frac{R(1+\sinb)}{1+\sinb}-\frac{R \sinb}{1+\sinb})(1+\cosb)(1+\sinb)} \\
		&= \frac{R \sinb}{(\frac{R(1+\sinb) - R\sinb}{1+\sinb})(1+\cosb)(1+\sinb)} \\
		&= \frac{R \sinb}{(\frac{R+R\sinb-R\sinb}{1+\sinb})(1+\cosb)(1+\sinb)} \\
		&= \frac{R \sinb}{(\frac{R}{1+\sinb})(1+\cosb)(1+\sinb)} \\
		&= \frac{\sinb}{(\frac{1}{1+\sinb})(1+\cosb)(1+\sinb)} \\
		&= \frac{(\sinb)(1+\sinb)}{(1+\sinb)(1+\cosb)} \\[15pt]
		m_{AE} &= \frac{\sinb}{1+\cosb}
	\end{align*}
\end{samepage}

\pagebreak
\subsection[]{Herleitung von $m_{CD}$}
Hierfür relevante Punkte und Strecken:

\begin{center}
	\begin{tikzpicture}[scale=4]
		% Punkte
		\Koordinaten
		\coordinate (c_) at (-0.5,0);
		\coordinate (d) at (-0.26864,0);
		\coordinate (kbm) at (-0.26864,0.46335);
		\coordinate (kbm_side) at (-0.5,0.46335);

		\filldraw[black] (m) circle (0.5pt) node[anchor=north]{$M$};
		\filldraw[black] (d) circle (0.5pt) node[anchor=north]{$D$};
		\filldraw[black] (p) circle (0.5pt) node[anchor=south west]{$P$};
		\filldraw[black] (c) circle (0.5pt) node[anchor=south east]{$C$};
		\filldraw[blue] (kbm) circle (0.35pt);

		\Halbkreis
		% Kreis k_b
		\draw (kbm) circle(0.46441);

		% Linien
		\draw[black] (m) -- (p) node[midway,right]{$R$};
		\draw[black] (c) -- (d);

		\draw[violet, line width=1pt] (m) -- (kbm) node[midway,above right]{$R-r$};
		\draw[magenta, line width=1pt] (kbm) -- (c) node[midway,above right]{$r$};
		\draw[blue, line width=1pt] (m) -- (d) node[midway,below]{$l$};
		\draw[orange, line width=1pt] (c_) -- (d) node[midway,below]{$x$};
		\draw[orange, dotted, line width=1pt] (kbm_side) -- (kbm) node[midway,below]{};
		\draw[magenta, line width=1pt] (c_) -- (kbm_side) node[midway,left]{$r$};
		\draw[magenta, dotted, line width=1pt] (d) -- (kbm) node[midway,left]{};
		\draw[teal, line width=1pt] (kbm_side) -- (c) node[midway, left]{$h$};

		% Winkel
		\pic [draw,"$\alpha$",angle eccentricity=1.7]{angle = p--m--c};
		\pic [draw,"$\alpha$",angle eccentricity=1.7]{angle = c--m--a};
		\pic [draw,angle eccentricity=1]{right angle = c--c_--m};
	\end{tikzpicture}
\end{center}
(Fall hier analog zur Herleitung von $m_{AE}$, siehe Erklärung oben)
\begin{align*}
	m_{CD} &= \frac{\Delta y}{\Delta x} = \frac{h+r}{-x}
\end{align*}

\begin{align*}
	\textcolor{violet}{(R-r)}^2 &= \textcolor{magenta}{r}^2+\textcolor{blue}{l}^2 \\
	l &= \sqrt{(R-r)^2-r^2} \\
	&= \sqrt{R(R-2r)} \tag{1}
\end{align*}

\begin{align*}
	\sina &= \frac{\text{Gegenkathete}}{\text{Hypotenuse}} \\
	\sina &= \frac{ \textcolor{magenta}{r} }{ \textcolor{violet}{R-r} } \\
	r &= R\sina - r\sina \\
	r(1+\sina) &= R\sina \\
	r &= \frac{R\sina}{1+\sina} \tag{2}
\end{align*}

\begin{align*}
	\textcolor{teal}{h}^2 + \textcolor{orange}{x}^2 &= \textcolor{magenta}{r}^2 \\
	h &= \sqrt{r^2-x^2} \tag{3}
\end{align*}

\begin{samepage}
	\begin{align*}
		\frac{l}{r} &= \frac{l+x}{r+h}
	\end{align*}
	Mit $(3)$:
	\begin{align*}
		\frac{l}{r} &= \frac{l+x}{r+\sqrt{r^2-x^2}} \\
		(r+\sqrt{r^2-x^2})l &= r(l+x) \\
		rl + l\sqrt{r^2-x^2} &= rl+rx \\
		l\sqrt{r^2-x^2} &= rx \\
		l^2(r^2-x^2) &= r^2 x^2 \\
		l^2 r^2 - l^2 x^2 &= r^2 x^2 \\
		r^2 x^2 + l^2 x^2 &= l^2 r^2 \\
		x^2 (r^2+l^2) &= l^2 r^2 \\
		x^2 &= \frac{l^2 r^2}{r^2 + l^2} \\
		x &= \frac{lr}{\sqrt{\textcolor{magenta}{r}^2+\textcolor{blue}{l}^2}} \\
		x &= \frac{lr}{\textcolor{violet}{R-r}}
	\end{align*}
	Mit $(1)$:
	\begin{align*}
		x &= \frac{\sqrt{R(R-2r)}r}{R-r}
	\end{align*}
	Mit $(2)$:
	\begin{align*}
		x &= \frac{\sqrt{R(R-2\frac{R\sina}{1+\sina})}\frac{R\sina}{1+\sina}}{R-\frac{R\sina}{1+\sina}} \\
		&= \frac{\sqrt{R(R-2R\frac{\sina}{1+\sina})}\frac{R\sina}{1+\sina}}{\frac{R}{1+\sina}} \\
		&= \sqrt{R\left(R-2R\frac{\sina}{1+\sina}\right)}\left(\frac{R\sina}{1+\sina}\right)\left(\frac{1+\sina}{R}\right) \\
		&= R\sqrt{1-2\frac{\sina}{1+\sina}} \, (\sina) \\
		&= R\sqrt{\frac{1+\sina}{1+\sina}-\frac{2\sina}{1+\sina}} \, (\sina) \\
		x &= R \, \sina \, \sqrt{\frac{1-\sina}{1+\sina}} \tag{4}
	\end{align*}
\end{samepage}
\goodbreak
\begin{samepage}
	Weitergehend von $(3)$, mit $(2)$ und $(4)$: \nopagebreak
	\begin{align*}
		h &= \sqrt{r^2-x^2} \\
		&= \sqrt{\left(\frac{R\sina}{1+\sina}\right)^2 - \left(R\sina\sqrt{\frac{1-\sina}{1+\sina}}\right)^2} \\
		&= \sqrt{\frac{(R\sina)^2}{(1+\sina)^2} - \frac{(R\sina)^2(1-\sina)}{1+\sina}} \\
		&= \sqrt{\frac{(R\sina)^2}{(1+\sina)^2} - \frac{(R\sina)^2(1-\sina)(1+\sina)}{(1+\sina)^2}} \\
		&= \frac{\sqrt{(R\sina)^2(1-(1-\sina)(1+\sina))}}{1+\sina} \\
		&= R\sina \, \frac{\sqrt{1-(1-\sina)(1+\sina)}}{1+\sina} \\
		&= R\sina \, \frac{\sqrt{1-(1-\sin^2\alpha)}}{1+\sina} \\
		&= R\sina \, \frac{\sqrt{\sin^2\alpha}}{1+\sina}
	\end{align*}
	Da $\alpha$ definitiv zwischen $\ang{0}$ und $\ang{180}$ liegt, gilt $\left|\sina\right| = \sina$:
	\begin{align*}
		h &= R\sina \, \frac{\sina}{1+\sina} \tag{$3^\prime$}
	\end{align*}
\end{samepage}
Mit $(2)$, $(3^\prime)$ und $(4)$: \nopagebreak
\begin{align*}
	m_{CD} &= \frac{h+r}{-x} \\
	&= \frac{R\sina \frac{\sina}{1+\sina} + \frac{R\sina}{1+\sina}}{-R\sina\sqrt{\frac{1-\sina}{1+\sina}}} \\
	&= \frac{\frac{\sina}{1+\sina}+\frac{1}{1+\sina}}{-\sqrt{\frac{1-\sina}{1+\sina}}} \\
	&= \frac{1}{-\sqrt{\frac{1-\sina}{1+\sina}}} \\
	&= -\sqrt{\frac{1+\sina}{1-\sina}} \\
	&= -\frac{\sqrt{1+\sina}}{\sqrt{1-\sina}} \\
	&= -\frac{\sqrt{1+\sina}\sqrt{1+\sina}}{\sqrt{1-\sina}\sqrt{1+\sina}} \\
	&= -\frac{1+\sina}{\sqrt{1 - \sin^2\alpha}} \\
	&= -\frac{\sina+1}{\sqrt{\cos^2\alpha}}
\end{align*}
Da unser $\alpha < \ang{90}$ ist, gilt $\cosb \geq \ang{0}$, also auch $\left|\cosa\right| = \cosa$:
\begin{align*}
	m_{CD} &= -\frac{\sina+1}{\cosa}
\end{align*}
\goodbreak
\begin{samepage}
	Für die Winkel $\alpha$ und $\beta$ gilt:
	\begin{center}
		\begin{tikzpicture}[scale=4]
			% Punkte
			\Koordinaten

			\filldraw[black] (m) circle (0.5pt) node[anchor=north]{$M$};
			\filldraw[black] (a) circle (0.5pt) node[anchor=north]{$A$};
			\filldraw[black] (b) circle (0.5pt) node[anchor=north]{$B$};
			\filldraw[black] (p) circle (0.5pt) node[anchor=south west]{$P$};
			\filldraw[black] (c) circle (0.5pt) node[anchor=south east]{$C$};
			\filldraw[black] (e) circle (0.5pt) node[anchor=south west]{$E$};

			\Halbkreis

			% Linien
			\draw[black] (m) -- (p);
			\draw[black] (m) -- (c);
			\draw[black] (m) -- (e);

			% Winkel
			\pic [draw,"$\alpha$",angle eccentricity=1.7]{angle = p--m--c};
			\pic [draw,"$\alpha$",angle eccentricity=1.7]{angle = c--m--a};
			\pic [draw,"$\beta$",angle eccentricity=1.7]{angle = b--m--e};
			\pic [draw,"$\beta$",angle eccentricity=1.7]{angle = e--m--p};
		\end{tikzpicture}
	\end{center}
	\begin{align*}
		2\beta + 2\alpha &= \ang{180} \\
		\alpha &= \ang{90} - \beta \\
		\sina &= \sin(\ang{90} - \beta) = \cosb \\
		\cosb &= \cos(\ang{90} - \beta) = \sinb \\
	\end{align*}
	\\
	So lässt sich $m_{CD}$ anders ausdrücken:
	\begin{align*}
		m_{CD} &= -\frac{\sina+1}{\cosa} \\
		&= -\frac{\cosb+1}{\sinb} \\[10pt]
		m_{AE} &= \frac{\sinb}{\cosb+1}
	\end{align*}
	\\
	Somit gilt $ m_{CD} = -\frac{1}{m_{AE}} $, und damit auch $CD \perp AE$. \qed
\end{samepage}

\pagebreak

\section{Aufgabe 4}
\setlength{\mathindent}{4cm}

% Was häufig benutzt wird
\newcommand{\Feld}{ +(-0.5,-0.5) rectangle +(0.5,0.5) }
\newcommand{\Ecke}{ [pattern={Lines[angle=135,distance=2]}, pattern color=orange] \Feld }

\newcommand{\RenateZug}{ [pattern={Lines[angle=45,distance=2]}, pattern color=magenta] \Feld }
\newcommand{\RenateFeld}{ [fill=magenta] \Feld }
\newcommand{\ErhardZug}{ [pattern={Lines[angle=45,distance=2]}, pattern color=teal] \Feld }
\newcommand{\ErhardFeld}{ [fill=teal] \Feld }
\newcommand{\UnbZug}{ [pattern={Lines[angle=45,distance=2]}, pattern color=gray] \Feld }
\newcommand{\UnbFeld}{ [fill=gray] \Feld }

% https://tex.stackexchange.com/a/98843
\newenvironment{NMCenter}{
	\vspace{\dimexpr-2\parsep-2\parskip\relax}
	\begin{center}
}{
	\end{center}
	\vspace{\dimexpr-2\parsep-2\parskip\relax}
}

In allen folgenden Diagrammen steht eine gestrichelte Linie für eine Symmetrieachse. Magenta und türkis markierte Felder sind ausgemalt, wobei der Anfänger die Farbe Magenta hat.
\\[10pt]
Läuft das Spiel zuletzt auf eine $2\times2$-Ecke hinaus, verliert der Spieler, der zuerst in dieses Feld malen muss:
\begin{center}
\begin{tikzpicture}[y=-0.5cm,x=0.5cm]
	\draw (1,1) \Feld;
	\draw (2,1) \Feld;
	\draw (2,2) \Feld;
	\draw [dashed] (3,0) -- (0,3);

	\node at (3.5,1.5) {$\rightarrow$};
	\draw (5,1) \RenateZug;
	\draw (6,1) \Feld;
	\draw (6,2) \Feld;
	\node at (7.5,1.5) {$\rightarrow$};
	\draw (9,1) \RenateFeld;
	\draw (10,1) \ErhardZug;
	\draw (10,2) \ErhardZug;
	\node[anchor=west] at (11.5,1.5) {$\rightarrow$ Anfänger verliert};

	\node at (3.5,4.5) {$\rightarrow$};
	\draw (5,4) \RenateZug;
	\draw (6,4) \RenateZug;
	\draw (6,5) \Feld;
	\node at (7.5,4.5) {$\rightarrow$};
	\draw (9,4) \RenateFeld;
	\draw (10,4) \RenateFeld;
	\draw (10,5) \ErhardZug;
	\node[anchor=west] at (11.5,4.5) {$\rightarrow$ Anfänger verliert};
\end{tikzpicture}
\end{center}
Basierend darauf gilt dies auch für eine Ecke, die in beide Dimensionen beliebig viele Felder größer ist. Derjenige, der nicht anfängt, kopiert exakt die Menge an ausgemalten Feldern des Anfängers ($q$) auf der anderen Kante und zwingt den anfangenden Spieler in eine weitere kleinere $p\times p$-Ecke (in Orange):
\begin{NMCenter}
\begin{tikzpicture}[y=-0.5cm,x=0.5cm]
	\draw (1,1) \RenateZug;
	\draw (2,1) \RenateZug;
	\node[anchor=center] at (3,1) {$...$};
	\draw (4,1) \RenateZug;
	\draw (5,1) \Ecke;
	\draw (5,2) \ErhardZug;
	\node[anchor=center] at (5,3) {$...$};
	\draw (5,4) \ErhardZug;
	\draw (5,5) \ErhardZug;
	\draw [dashed] (6,0) -- (3,3);

	\draw [decorate,decoration={brace,amplitude=5pt}] (0.55,0) -- (4.45,0) node[midway,yshift=1em]{$q$};
	\draw [decorate,decoration={brace,amplitude=5pt}] (4.55,0) -- (5.45,0) node[midway,yshift=1em]{$p$};

	\draw [decorate,decoration={brace,amplitude=5pt}] (6,0.55) -- (6,1.45) node[midway,xshift=1em]{$p$};
	\draw [decorate,decoration={brace,amplitude=5pt}] (6,1.55) -- (6,5.45) node[midway,xshift=1em]{$q$};
\end{tikzpicture}
\end{NMCenter}
Fährt Renate diese Strategie, kann sie Erhard irgendwann in eine $2\times2$-Ecke zwingen, in der er verliert, er hat also dann sicher verloren, wenn er in einer $p\times p$-Ecke beginnen muss (für ein beliebiges $p \geq 2$).
\\[10pt]
Läuft das Spiel auf eine \say{Hufeisenform} hinaus, verliert auch wieder der Spieler, der zuerst innerhalb dieser Form dran ist:
\\
\begin{NMCenter}
\begin{tikzpicture}[y=-0.45cm,x=0.45cm]
	\draw (1,1) \Feld;
	\draw (2,1) \Feld;
	\draw (3,1) \Feld;
	\draw (4,1) \Feld;
	\draw (5,1) \Feld;
	\draw (6,1) \Feld;
	\draw (6,2) \Feld;
	\draw (6,3) \Feld;

	\draw (6,4) \ErhardZug;
	\draw (1,4) \RenateZug;

	\draw (1,3) \Feld;
	\draw (1,2) \Feld;

	\draw [dashed] (3.5,0) -- (3.5,5);
	\node[anchor=west] at (6.5,2.5) {$\rightarrow$};
\end{tikzpicture}
\begin{tikzpicture}[y=-0.45cm,x=0.45cm]
	\draw (1,1) \Feld;
	\draw (2,1) \Feld;
	\draw (3,1) \Feld;
	\draw (4,1) \Feld;
	\draw (5,1) \Feld;
	\draw (6,1) \Feld;
	\draw (6,2) \Feld;
	\draw (6,3) \RenateZug;

	\draw (6,4) \ErhardFeld;
	\draw (1,4) \RenateFeld;

	\draw (1,3) \ErhardZug;
	\draw (1,2) \Feld;

	\draw [dashed] (3.5,0) -- (3.5,5);
	\node[anchor=west] at (6.5,2.5) {$\rightarrow$};
\end{tikzpicture}
\begin{tikzpicture}[y=-0.45cm,x=0.45cm]
	\draw (1,1) \ErhardZug;
	\draw (2,1) \ErhardZug;
	\draw (3,1) \ErhardZug;
	\draw (4,1) \ErhardZug;
	\draw (5,1) \Ecke;
	\draw (6,1) \Ecke;
	\draw (6,2) \Ecke;
	\draw (6,3) \RenateFeld;

	\draw (6,4) \ErhardFeld;
	\draw (1,4) \RenateFeld;

	\draw (1,3) \ErhardFeld;
	\draw (1,2) \RenateZug;

	\draw [dashed] (3.5,0) -- (3.5,5);
	\node[anchor=west] at (6.5,2.5) {\,oder};
\end{tikzpicture}
\begin{tikzpicture}[y=-0.45cm,x=0.45cm]
	\draw (1,1) \RenateZug;
	\draw (2,1) \ErhardZug;
	\draw (3,1) \ErhardZug;
	\draw (4,1) \ErhardZug;
	\draw (5,1) \Ecke;
	\draw (6,1) \Ecke;
	\draw (6,2) \Ecke;
	\draw (6,3) \RenateFeld;

	\draw (6,4) \ErhardFeld;
	\draw (1,4) \RenateFeld;

	\draw (1,3) \ErhardFeld;
	\draw (1,2) \RenateZug;

	\draw [dashed] (3.5,0) -- (3.5,5);
\end{tikzpicture}
\end{NMCenter}
Erhard (magenta) muss anfangen und Renate (türkis) kopiert solange seine Spielzüge auf der jeweils anderen Seite der Spiegelachse (welche die längere Kante halbiert), bis Erhard zum ersten Mal ein Feld ausmalt, das die längere Kante berührt oder in dieser liegt. Dann kann Renate Erhard in eine $p\times p$-Ecke zwingen und Renate gewinnt (wie oben bewiesen).
\pagebreak

Renate hat nun für zwei verschiedene Arten von Feldern Strategien:
\begin{itemize}
	\item[Fall 1:] $m=n$
\end{itemize}

Hierfür beginnt Renate mit dem Ausmalen eines einzelnen Feldes E in einer Ecke. Erhard kann in seinem nächsten Zug nur bis zu $n-1$ Felder auf genau einer Kante ausmalen. Renate wird solange mit jedem ihrer Züge das Spielfeld symmetrisch zur Spiegelachse, die die Strecke zwischen E und der Ecke diagonal gegenüber von E ist (eingezeichnet als gestrichelte Linie), halten, bis sich Erhard als Anfänger in einer $p\times p$-Ecke findet und verliert.
\\[10pt]
Ab dem zweiten Diagramm macht pro Bild immer erst Erhard (türkis) einen Zug, dann Renate (magenta):
\begin{NMCenter}
\begin{tikzpicture}[y=-0.5cm,x=0.5cm]
	\draw (1,1) \Feld;
	\draw (2,1) \Feld;
	\draw (3,1) \Feld;
	\draw (4,1) \Feld;
	\draw (5,1) \Feld;
	\draw (6,1) \Feld;
	\draw (6,2) \Feld;
	\draw (6,3) \Feld;
	\draw (6,4) \Feld;
	\draw (6,5) \Feld;
	\draw (6,6) \Feld;

	\draw (5,6) \Feld;
	\draw (4,6) \Feld;
	\draw (3,6) \Feld;
	\draw (2,6) \Feld;
	\draw (1,6) \RenateZug;

	\draw (1,5) \Feld;
	\draw (1,4) \Feld;
	\draw (1,3) \Feld;
	\draw (1,2) \Feld;

	\draw [dashed] (7,0) -- (0,7);
	\node[anchor=west] at (7.1,3.5) {$\rightarrow$};
\end{tikzpicture}
\begin{tikzpicture}[y=-0.5cm,x=0.5cm]
	\draw (1,1) \Feld;
	\draw (2,1) \Feld;
	\draw (3,1) \Feld;
	\draw (4,1) \Feld;
	\draw (5,1) \Feld;
	\draw (6,1) \Feld;
	\draw (6,2) \Feld;
	\draw (6,3) \Feld;
	\draw (6,4) \Feld;
	\draw (6,5) \Feld;
	\draw (6,6) \Feld;

	\draw (5,6) \Feld;
	\draw (4,6) \Feld;
	\draw (3,6) \Feld;
	\draw (2,6) \ErhardZug;
	\draw (1,6) \RenateFeld;

	\draw (1,5) \RenateZug;
	\draw (1,4) \Feld;
	\draw (1,3) \Feld;
	\draw (1,2) \Feld;

	\draw [dashed] (7,0) -- (0,7);
	\node[anchor=west] at (7.1,3.5) {$\rightarrow$};
\end{tikzpicture}
\begin{tikzpicture}[y=-0.5cm,x=0.5cm]
	\draw (1,1) \Feld;
	\draw (2,1) \Feld;
	\draw (3,1) \Feld;
	\draw (4,1) \Feld;
	\draw (5,1) \Feld;
	\draw (6,1) \Feld;
	\draw (6,2) \Feld;
	\draw (6,3) \Feld;
	\draw (6,4) \Feld;
	\draw (6,5) \Feld;
	\draw (6,6) \Feld;

	\draw (5,6) \Feld;
	\draw (4,6) \RenateZug;
	\draw (3,6) \RenateZug;
	\draw (2,6) \ErhardFeld;
	\draw (1,6) \RenateFeld;

	\draw (1,5) \RenateFeld;
	\draw (1,4) \ErhardZug;
	\draw (1,3) \ErhardZug;
	\draw (1,2) \Feld;

	\draw [dashed] (7,0) -- (0,7);
\end{tikzpicture}
\end{NMCenter}
Egal wieviele Felder Erhard in seinem nächsten Zug ausmalt, wird er in eine $p\times p$-Ecke (gezeichnet in Orange) gezwungen und verliert:
\\
\begin{NMCenter}
\begin{tikzpicture}[y=-0.5cm,x=0.5cm]
	\draw (1,1) \Ecke;
	\draw (2,1) \Ecke;
	\draw (3,1) \Ecke;
	\draw (4,1) \Ecke;
	\draw (5,1) \Ecke;
	\draw (6,1) \Ecke;
	\draw (6,2) \Ecke;
	\draw (6,3) \Ecke;
	\draw (6,4) \Ecke;
	\draw (6,5) \Ecke;
	\draw (6,6) \Ecke;

	\draw (5,6) \ErhardZug;
	\draw (4,6) \RenateFeld;
	\draw (3,6) \RenateFeld;
	\draw (2,6) \ErhardFeld;
	\draw (1,6) \RenateFeld;

	\draw (1,5) \RenateFeld;
	\draw (1,4) \ErhardFeld;
	\draw (1,3) \ErhardFeld;
	\draw (1,2) \RenateZug;

	\draw [dashed] (7,0) -- (0,7);
	\node[anchor=west] at (7.1,3.5) {oder};
\end{tikzpicture}
\begin{tikzpicture}[y=-0.5cm,x=0.5cm]
	\draw (1,1) \RenateZug;
	\draw (2,1) \Ecke;
	\draw (3,1) \Ecke;
	\draw (4,1) \Ecke;
	\draw (5,1) \Ecke;
	\draw (6,1) \Ecke;
	\draw (6,2) \Ecke;
	\draw (6,3) \Ecke;
	\draw (6,4) \Ecke;
	\draw (6,5) \Ecke;
	\draw (6,6) \ErhardZug;

	\draw (5,6) \ErhardZug;
	\draw (4,6) \RenateFeld;
	\draw (3,6) \RenateFeld;
	\draw (2,6) \ErhardFeld;
	\draw (1,6) \RenateFeld;

	\draw (1,5) \RenateFeld;
	\draw (1,4) \ErhardFeld;
	\draw (1,3) \ErhardFeld;
	\draw (1,2) \RenateZug;

	\draw [dashed] (7,0) -- (0,7);
\end{tikzpicture}
\end{NMCenter}

\begin{itemize}
	\item[Fall 2:] $m\neq n$
\end{itemize}

Hierbei malt Renate in ihrem ersten Zug eine der längeren Kanten komplett aus. So bleibt als Spielfeld noch eine \say{Hufeisenform} übrig. Nach dem Beweis oben verliert der Anfänger (nach Renates Strategie also Erhard) in diesem Hufeisen auch (wie bereits oben gezeigt), Renate hat also hier auch eine sichere Gewinnstrategie (grau steht hier für beliebige Züge beider Seiten):
\\
\begin{NMCenter}
\begin{tikzpicture}[y=-0.5cm,x=0.5cm]
	\draw (1,1) \Feld;
	\draw (2,1) \Feld;
	\draw (3,1) \Feld;
	\draw (4,1) \Feld;
	\draw (5,1) \Feld;
	\draw (6,1) \Feld;
	\draw (7,1) \Feld;
	\draw (8,1) \Feld;

	\draw (8,2) \Feld;
	\draw (8,3) \Feld;
	\draw (8,4) \Feld;
	\draw (8,5) \Feld;

	\draw (8,6) \RenateZug;
	\draw (7,6) \RenateZug;
	\draw (6,6) \RenateZug;
	\draw (5,6) \RenateZug;
	\draw (4,6) \RenateZug;
	\draw (3,6) \RenateZug;
	\draw (2,6) \RenateZug;
	\draw (1,6) \RenateZug;

	\draw (1,5) \Feld;
	\draw (1,4) \Feld;
	\draw (1,3) \Feld;
	\draw (1,2) \Feld;

	\draw [dashed] (4.5,0) -- (4.5,7);
	\node[anchor=west] at (8.7,3.5) {$\rightarrow$};
\end{tikzpicture}
\begin{tikzpicture}[y=-0.5cm,x=0.5cm]
	\draw (1,1) \Feld;
	\draw (2,1) \Feld;
	\draw (3,1) \Feld;
	\draw (4,1) \Feld;
	\draw (5,1) \Feld;
	\draw (6,1) \Feld;
	\draw (7,1) \Feld;
	\draw (8,1) \Feld;

	\draw (8,2) \UnbZug;
	\draw (8,3) \UnbZug;
	\draw (8,4) \UnbZug;
	\draw (8,5) \UnbZug;

	\draw (8,6) \RenateFeld;
	\draw (7,6) \RenateFeld;
	\draw (6,6) \RenateFeld;
	\draw (5,6) \RenateFeld;
	\draw (4,6) \RenateFeld;
	\draw (3,6) \RenateFeld;
	\draw (2,6) \RenateFeld;
	\draw (1,6) \RenateFeld;

	\draw (1,5) \UnbZug;
	\draw (1,4) \Feld;
	\draw (1,3) \Feld;
	\draw (1,2) \Feld;

	\draw [dashed] (4.5,0) -- (4.5,7);
	\node[anchor=west] at (8.7,3.5) {$\rightarrow$};
\end{tikzpicture}
\begin{tikzpicture}[y=-0.5cm,x=0.5cm]
	\draw (1,1) \Ecke;
	\draw (2,1) \Ecke;
	\draw (3,1) \Ecke;
	\draw (4,1) \Ecke;
	\draw (5,1) \RenateZug;
	\draw (6,1) \RenateZug;
	\draw (7,1) \RenateZug;
	\draw (8,1) \RenateZug;

	\draw (8,2) \UnbFeld;
	\draw (8,3) \UnbFeld;
	\draw (8,4) \UnbFeld;
	\draw (8,5) \UnbFeld;

	\draw (8,6) \RenateFeld;
	\draw (7,6) \RenateFeld;
	\draw (6,6) \RenateFeld;
	\draw (5,6) \RenateFeld;
	\draw (4,6) \RenateFeld;
	\draw (3,6) \RenateFeld;
	\draw (2,6) \RenateFeld;
	\draw (1,6) \RenateFeld;

	\draw (1,5) \UnbFeld;
	\draw (1,4) \Ecke;
	\draw (1,3) \Ecke;
	\draw (1,2) \Ecke;

	\draw [dashed] (4.5,0) -- (4.5,7);
\end{tikzpicture}
\end{NMCenter}

Renate hat also durch Fall 1 und 2 eine Gewinnstrategie für alle erlaubten Spielfeldgrößen, also für alle $m, n \geq 3$. \qed


\end{document}
