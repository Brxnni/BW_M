\setlength{\mathindent}{4cm}

% Was häufig benutzt wird
\newcommand{\Koordinaten}{
	\coordinate (m) at (0,0);
	\coordinate (a) at (-1,0);
	\coordinate (b) at (1,0);
	\coordinate (c) at (-0.5,0.866);
	\coordinate (p) at (0.5,0.866);
	\coordinate (e) at (0.866,0.5);
}
\newcommand{\Halbkreis}{
	\begin{scope}
		\clip (a) rectangle (1,1);
		\draw[black] (m) circle(1);
		\draw[black] (a) -- (b);
	\end{scope}
}

\newcommand{\Strecke}[1]{\overline{\mathit{#1}}}

\newcommand{\sina}{\sin\alpha}
\newcommand{\cosa}{\cos\alpha}
\newcommand{\sinb}{\sin\beta}
\newcommand{\cosb}{\cos\beta}

Um zu beweisen, dass $AE$ senkrecht zu $CD$ steht, werden die Steigungen beider Geraden in Abhängigkeit des Winkels $\beta$, der die Position von P auf dem Halbkreisbogen $AB$ bestimmt, berechnet.
\\[10pt]
(Für die gesamte Herleitung beziehen sich, wenn nicht durch Klammern anders angegeben, $\sin$ und $\cos$ immer nur auf die Variable unmittelbar danach, also nur $\alpha$ oder $\beta$.)

\subsection{Herleitung von $m_{AE}$}
Hierfür relevante Punkte und Strecken:
\begin{center}
	\begin{tikzpicture}[scale=4]
		% Koordinaten
		\Koordinaten
		\coordinate (kam) at (0.57732,0.33335);

		\Halbkreis
		% Kreis k_a
		\draw (kam) circle(0.33334);

		% Punkte
		\filldraw[black] (m) circle (0.5pt) node[anchor=north]{$M$};
		\filldraw[black] (a) circle (0.5pt) node[anchor=north]{$A$};
		\filldraw[black] (b) circle (0.5pt) node[anchor=north]{$B$};
		\filldraw[black] (p) circle (0.5pt) node[anchor=south west]{$P$};
		\filldraw[black] (e) circle (0.5pt) node[anchor=south west]{$E$};
		\filldraw[blue] (kam) circle (0.35pt);

		% Linien
		\draw[black] (a) -- (e);
		\draw[black] (m) -- (p);
		\draw[magenta, line width=1pt] (m) -- (e) node[midway,above]{$R$};
		\draw[magenta, line width=1pt] (a) -- (m) node[midway,below]{$R$};
		\draw[blue] (kam) -- (e) node[midway,below right]{$r$};
		\draw[blue] (kam) -- (0.57732,0) node[midway,right]{$r$};
		\draw[orange, line width=1pt] (e) -- (0.866,0) node[midway,left]{$h$};
		\draw[teal, line width=1pt] (b) -- (0.866,0) node[midway,below]{$x$};

		% Winkel
		\pic [draw,"$\beta$",angle eccentricity=1.7]{angle = b--m--e};
		\pic [draw,"$\beta$",angle eccentricity=1.7]{angle = e--m--p};
	\end{tikzpicture}
\end{center}
% diese erklärung ist nicht gut das muss besser irgendwie besser gehen (auch egal weil es die beim bwm eh nicht zu sehen bekommen, gibt glaub ich auch punktabzug :c)
Die Strecke $\Strecke{ME}$ halbiert exakt den Kreissektor, weil der Mittelpunkt des Kreises $k_a$ exakt zwischen $M$ und $E$ liegt, also auf $\Strecke{ME}$. Das liegt wiederum daran, dass $k_a$ maximal groß sein soll, aber $\Strecke{MB}$ und $\Strecke{MP}$ in genau einem Punkt berühren soll. Um maximal groß zu sein, \say{wächst} er also symmetrisch zu der Winkelhalbierenden in seinem Kreissektor, also symmetrisch zu der Strecke $\Strecke{ME}$.
\begin{align*}
	m_{AE} &= \frac{\Delta y}{\Delta x} = \frac{h}{2R-x} \\
	\frac{h}{R} = \frac{r}{R-r} \leftrightarrow h &= \frac{Rr}{R-r} \tag{1}
\end{align*}

\begin{align*}
	\sinb &= \frac{\text{Gegenkathete}}{\text{Hypotenuse}} \\
	\sinb &= \frac{r}{R-r} \\
	r &= R \sinb - r \sinb \\
	r + r \sinb &= R \sinb \\
	r(1+\sinb) &= R \sinb \\
	r &= \frac{R \sinb}{1 + \sinb} \tag{2}
\end{align*}

\pagebreak
\begin{center}
	\begin{tikzpicture}[scale=13]
		% Punkte
		\Koordinaten
		\coordinate (b2) at (0.93969,0.34202);
		\coordinate (p2) at (0.98481,0.17365);
		\coordinate (p2_) at (0.98481,0);
		\coordinate (kam) at (0.83896,0.14772);
		\coordinate (kam_side) at (0.98481,0.14772);

		\filldraw[black] (m) circle (0.25pt) node[anchor=east]{$M$};
		\filldraw[black] (p2) circle (0.25pt) node[anchor=south west]{$P$};

		% Kreissektor mit 20°
		\draw[black] (m) -- (0:1) arc(0:20:1) -- cycle;
		% Kreis k_a
		\draw (kam) circle(0.14812);

		% Linien
		\draw[orange, line width=1pt] (p2) -- (0.98481,0) node[midway,right]{$h$};
		\draw[magenta, line width=1pt] (m) -- (b2) node[midway,above left]{$R$};
		\draw[violet] (kam) -- (p2) node[midway,above]{$r$};
		\draw[violet, line width=1pt] (kam) -- (0.83896,0) node[midway,left]{$r$};
		\draw[blue, line width=1pt] (m) -- (p2_) node[midway,below]{$R_0$};
		\draw[teal, line width=1pt] (p2_) -- (b) node[midway,below]{$x$};
		\draw[cyan, line width=1pt] (kam) -- (kam_side) node[midway,below]{$b$};
		\draw[black] (m) -- (p2);

		% Winkel
		\pic [draw,"$\beta$",angle eccentricity=2.6]{angle = b--m--b2};
		\pic [draw,"\tiny$\beta$\normalsize",angle eccentricity=2]{angle = kam_side--kam--p2};
	\end{tikzpicture}
\end{center}

\begin{align*}
	\frac{R_0}{h} &= \frac{R_0-b}{r} \\
	r R_0 &= h R_0 - hb \\
	r R_0 - h R_0 &= -hb \\
	R_0 (r-h) &= -hb \\
	R_0 &= \frac{-hb}{r-h}
\end{align*}
Mit $ \cosb = \frac{b}{r} \Leftrightarrow b = r \cosb $ und $(1)$:
\begin{align*}
	R_0 &= \frac{\frac{-Rr}{R-r}*r \cosb}{r-\frac{Rr}{R-r}} \\
	&= \frac{\left(\frac{-Rr}{R-r}*r \cosb\right)}{\left(\frac{(R-r)r}{R-r}-\frac{Rr}{R-r}\right)} \\
	&= \frac{\left(\frac{-Rr}{R-r}*r \cosb\right)}{\left(\frac{Rr-r^2-Rr}{R-r}\right)} \\
	&= \frac{\left(\frac{-Rr^2 \cosb}{R-r}\right)}{\left(-\frac{r^2}{R-r}\right)} \\
	&= \frac{-Rr^2 \cosb * (R-r)}{-(R-r)r^2} \\
	R_0 &= R \cosb
\end{align*}

\begin{align*}
	R_0 + x &= R \\
	x &= R - R_0 \\
	&= R - R \cosb \tag{3}
\end{align*}

\begin{samepage}
	\begin{align*}
		m_{AE} &= \frac{h}{2R - x} \\
	\end{align*}
	Mit $(1)$ und $(3)$: \nopagebreak
	\begin{align*}
		m_{AE} &= \frac{\frac{Rr}{R-r}}{2R - (R-R \cosb)} \\
		&= \frac{\frac{Rr}{R-r}}{R+R \cosb} \\
		&= \frac{Rr}{(R-r)(R+R \cosb)} \\
		&= \frac{r}{(R-r)(1+\cosb)}
	\end{align*}
	Mit $(2)$: \nopagebreak
	\begin{align*}
		m_{AE} &= \frac{\frac{R \sinb}{1+\sinb}}{(R-r)(1+\cosb)} \\
		&= \frac{R \sinb}{(R-r)(1+\cosb)(1+\sinb)} \\
		&= \frac{R \sinb}{(R-\frac{R \sinb}{1+\sinb})(1+\cosb)(1+\sinb)} \\
		&= \frac{R \sinb}{(\frac{R(1+\sinb)}{1+\sinb}-\frac{R \sinb}{1+\sinb})(1+\cosb)(1+\sinb)} \\
		&= \frac{R \sinb}{(\frac{R(1+\sinb) - R\sinb}{1+\sinb})(1+\cosb)(1+\sinb)} \\
		&= \frac{R \sinb}{(\frac{R+R\sinb-R\sinb}{1+\sinb})(1+\cosb)(1+\sinb)} \\
		&= \frac{R \sinb}{(\frac{R}{1+\sinb})(1+\cosb)(1+\sinb)} \\
		&= \frac{\sinb}{(\frac{1}{1+\sinb})(1+\cosb)(1+\sinb)} \\
		&= \frac{(\sinb)(1+\sinb)}{(1+\sinb)(1+\cosb)} \\[15pt]
		m_{AE} &= \frac{\sinb}{1+\cosb}
	\end{align*}
\end{samepage}

\pagebreak
\subsection{Herleitung von $m_{CD}$}
Hierfür relevante Punkte und Strecken:

\begin{center}
	\begin{tikzpicture}[scale=4]
		% Punkte
		\Koordinaten
		\coordinate (c_) at (-0.5,0);
		\coordinate (d) at (-0.26864,0);
		\coordinate (kbm) at (-0.26864,0.46335);
		\coordinate (kbm_side) at (-0.5,0.46335);

		\filldraw[black] (m) circle (0.5pt) node[anchor=north]{$M$};
		\filldraw[black] (d) circle (0.5pt) node[anchor=north]{$D$};
		\filldraw[black] (p) circle (0.5pt) node[anchor=south west]{$P$};
		\filldraw[black] (c) circle (0.5pt) node[anchor=south east]{$C$};
		\filldraw[blue] (kbm) circle (0.35pt);

		\Halbkreis
		% Kreis k_b
		\draw (kbm) circle(0.46441);

		% Linien
		\draw[black] (m) -- (p) node[midway,right]{$R$};
		\draw[black] (c) -- (d);

		\draw[violet, line width=1pt] (m) -- (kbm) node[midway,above right]{$R-r$};
		\draw[magenta, line width=1pt] (kbm) -- (c) node[midway,above right]{$r$};
		\draw[blue, line width=1pt] (m) -- (d) node[midway,below]{$l$};
		\draw[orange, line width=1pt] (c_) -- (d) node[midway,below]{$x$};
		\draw[orange, dotted, line width=1pt] (kbm_side) -- (kbm) node[midway,below]{};
		\draw[magenta, line width=1pt] (c_) -- (kbm_side) node[midway,left]{$r$};
		\draw[magenta, dotted, line width=1pt] (d) -- (kbm) node[midway,left]{};
		\draw[teal, line width=1pt] (kbm_side) -- (c) node[midway, left]{$h$};

		% Winkel
		\pic [draw,"$\alpha$",angle eccentricity=1.7]{angle = p--m--c};
		\pic [draw,"$\alpha$",angle eccentricity=1.7]{angle = c--m--a};
		\pic [draw,angle eccentricity=1]{right angle = c--c_--m};
	\end{tikzpicture}
\end{center}
(Fall hier analog zur Herleitung von $m_{AE}$, siehe Erklärung oben)
\begin{align*}
	m_{CD} &= \frac{\Delta y}{\Delta x} = \frac{h+r}{-x}
\end{align*}

\begin{align*}
	\textcolor{violet}{(R-r)}^2 &= \textcolor{magenta}{r}^2+\textcolor{blue}{l}^2 \\
	l &= \sqrt{(R-r)^2-r^2} \\
	&= \sqrt{R(R-2r)} \tag{1}
\end{align*}

\begin{align*}
	\sina &= \frac{\text{Gegenkathete}}{\text{Hypotenuse}} \\
	\sina &= \frac{ \textcolor{magenta}{r} }{ \textcolor{violet}{R-r} } \\
	r &= R\sina - r\sina \\
	r(1+\sina) &= R\sina \\
	r &= \frac{R\sina}{1+\sina} \tag{2}
\end{align*}

\begin{align*}
	\textcolor{teal}{h}^2 + \textcolor{orange}{x}^2 &= \textcolor{magenta}{r}^2 \\
	h &= \sqrt{r^2-x^2} \tag{3}
\end{align*}

\begin{samepage}
	\begin{align*}
		\frac{l}{r} &= \frac{l+x}{r+h}
	\end{align*}
	Mit $(3)$:
	\begin{align*}
		\frac{l}{r} &= \frac{l+x}{r+\sqrt{r^2-x^2}} \\
		(r+\sqrt{r^2-x^2})l &= r(l+x) \\
		rl + l\sqrt{r^2-x^2} &= rl+rx \\
		l\sqrt{r^2-x^2} &= rx \\
		l^2(r^2-x^2) &= r^2 x^2 \\
		l^2 r^2 - l^2 x^2 &= r^2 x^2 \\
		r^2 x^2 + l^2 x^2 &= l^2 r^2 \\
		x^2 (r^2+l^2) &= l^2 r^2 \\
		x^2 &= \frac{l^2 r^2}{r^2 + l^2} \\
		x &= \frac{lr}{\sqrt{\textcolor{magenta}{r}^2+\textcolor{blue}{l}^2}} \\
		x &= \frac{lr}{\textcolor{violet}{R-r}}
	\end{align*}
	Mit $(1)$:
	\begin{align*}
		x &= \frac{\sqrt{R(R-2r)}r}{R-r}
	\end{align*}
	Mit $(2)$:
	\begin{align*}
		x &= \frac{\sqrt{R(R-2\frac{R\sina}{1+\sina})}\frac{R\sina}{1+\sina}}{R-\frac{R\sina}{1+\sina}} \\
		&= \frac{\sqrt{R(R-2R\frac{\sina}{1+\sina})}\frac{R\sina}{1+\sina}}{\frac{R}{1+\sina}} \\
		&= \sqrt{R\left(R-2R\frac{\sina}{1+\sina}\right)}\left(\frac{R\sina}{1+\sina}\right)\left(\frac{1+\sina}{R}\right) \\
		&= R\sqrt{1-2\frac{\sina}{1+\sina}} \, (\sina) \\
		&= R\sqrt{\frac{1+\sina}{1+\sina}-\frac{2\sina}{1+\sina}} \, (\sina) \\
		x &= R \, \sina \, \sqrt{\frac{1-\sina}{1+\sina}} \tag{4}
	\end{align*}
\end{samepage}
\goodbreak
\begin{samepage}
	Weitergehend von $(3)$, mit $(2)$ und $(4)$: \nopagebreak
	\begin{align*}
		h &= \sqrt{r^2-x^2} \\
		&= \sqrt{\left(\frac{R\sina}{1+\sina}\right)^2 - \left(R\sina\sqrt{\frac{1-\sina}{1+\sina}}\right)^2} \\
		&= \sqrt{\frac{(R\sina)^2}{(1+\sina)^2} - \frac{(R\sina)^2(1-\sina)}{1+\sina}} \\
		&= \sqrt{\frac{(R\sina)^2}{(1+\sina)^2} - \frac{(R\sina)^2(1-\sina)(1+\sina)}{(1+\sina)^2}} \\
		&= \frac{\sqrt{(R\sina)^2(1-(1-\sina)(1+\sina))}}{1+\sina} \\
		&= R\sina \, \frac{\sqrt{1-(1-\sina)(1+\sina)}}{1+\sina} \\
		&= R\sina \, \frac{\sqrt{1-(1-\sin^2\alpha)}}{1+\sina} \\
		&= R\sina \, \frac{\sqrt{\sin^2\alpha}}{1+\sina}
	\end{align*}
	Da $\alpha$ definitiv zwischen $\ang{0}$ und $\ang{180}$ liegt, gilt $\left|\sina\right| = \sina$:
	\begin{align*}
		h &= R\sina \, \frac{\sina}{1+\sina} \tag{$3^\prime$}
	\end{align*}
\end{samepage}
Mit $(2)$, $(3^\prime)$ und $(4)$: \nopagebreak
\begin{align*}
	m_{CD} &= \frac{h+r}{-x} \\
	&= \frac{R\sina \frac{\sina}{1+\sina} + \frac{R\sina}{1+\sina}}{-R\sina\sqrt{\frac{1-\sina}{1+\sina}}} \\
	&= \frac{\frac{\sina}{1+\sina}+\frac{1}{1+\sina}}{-\sqrt{\frac{1-\sina}{1+\sina}}} \\
	&= \frac{1}{-\sqrt{\frac{1-\sina}{1+\sina}}} \\
	&= -\sqrt{\frac{1+\sina}{1-\sina}} \\
	&= -\frac{\sqrt{1+\sina}}{\sqrt{1-\sina}} \\
	&= -\frac{\sqrt{1+\sina}\sqrt{1+\sina}}{\sqrt{1-\sina}\sqrt{1+\sina}} \\
	&= -\frac{1+\sina}{\sqrt{1 - \sin^2\alpha}} \\
	&= -\frac{\sina+1}{\sqrt{\cos^2\alpha}}
\end{align*}
Da $\alpha \leq \ang{90}$ gilt, folgt analog zu zuvor $\left|\cosa\right| = \cosa$:
\begin{align*}
	m_{CD} &= -\frac{\sina+1}{\cosa}
\end{align*}
\goodbreak
\begin{samepage}
	Für die Winkel $\alpha$ und $\beta$ gilt:
	\begin{center}
		\begin{tikzpicture}[scale=4]
			% Punkte
			\Koordinaten

			\filldraw[black] (m) circle (0.5pt) node[anchor=north]{$M$};
			\filldraw[black] (a) circle (0.5pt) node[anchor=north]{$A$};
			\filldraw[black] (b) circle (0.5pt) node[anchor=north]{$B$};
			\filldraw[black] (p) circle (0.5pt) node[anchor=south west]{$P$};
			\filldraw[black] (c) circle (0.5pt) node[anchor=south east]{$C$};
			\filldraw[black] (e) circle (0.5pt) node[anchor=south west]{$E$};

			\Halbkreis

			% Linien
			\draw[black] (m) -- (p);
			\draw[black] (m) -- (c);
			\draw[black] (m) -- (e);

			% Winkel
			\pic [draw,"$\alpha$",angle eccentricity=1.7]{angle = p--m--c};
			\pic [draw,"$\alpha$",angle eccentricity=1.7]{angle = c--m--a};
			\pic [draw,"$\beta$",angle eccentricity=1.7]{angle = b--m--e};
			\pic [draw,"$\beta$",angle eccentricity=1.7]{angle = e--m--p};
		\end{tikzpicture}
	\end{center}
	\begin{align*}
		2\alpha + 2\beta &= \ang{180} \\
		\alpha &= \ang{90} - \beta \\
		\sina &= \sin(\ang{90} - \beta) = \cosb \\
		\cosa &= \cos(\ang{90} - \beta) = \sinb \\
	\end{align*}
	\\
	So lässt sich $m_{CD}$ anders ausdrücken:
	\begin{align*}
		m_{CD} &= -\frac{\sina+1}{\cosa} \\
		&= -\frac{\cosb+1}{\sinb} \\[10pt]
		m_{AE} &= \frac{\sinb}{\cosb+1}
	\end{align*}
	\\
	Somit gilt $ m_{CD} = -\frac{1}{m_{AE}} $, und damit auch $CD \perp AE$. \qed
\end{samepage}
